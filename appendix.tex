% # -*- coding: utf-8 -*-
\typeout{now loading appendix.tex}
\appendix
\pagenumbering{Roman}
\section{附录}
\subsection{说明}
-------------------------------------修订历史----------------------------------\\
Version1\\
2011-05-05修改说明文部分\\%扫描,老师仅重新修订了说明文部分,
Version2\\
2011-05-06修改部分错误,补充说明文部分,添加目录\\%修改部分错误,根据老师上课讲的内容补充说明文部分(主要是整体感知中的路标性词语),并增加“目录”内容
2011-05-15添加“线索”\\%高级添加(线索、文言文翻译)
2011-05-19修改说明文、议论文部分\\%五、1“说明文整体感知”细微调整,说明文和议论文部分的几个括号的修正
Version3\\
2011-05-20补充、修改一、四部分,修改二部分\\%一、4改为大括号,补充四、1“议论文整体感知”,补充一、8“论据的种类”,修改议论文部分(主要是论据种类、论证方法、论证结构以及他们的答题格式),修改二、19“语言表现力(赏析)”,删除文言文部分
Version3.1\\
2011-05-24修改%从学校偷来XZHU盘全部课件,并增加了文言文部分\\
2011-05-24-2011-05-25增加修改标记%并复刻原稿和原稿plus(即加上xzh亲笔来源于2011-05-05)此次修改标记的添加规则如下:\\
%以原稿plus为ver0,进行一一对比(下称0稿)\\
添加字符由三个部分组成:\\
时间、来源、版本\\
%版本:此次版本号凡是被修改的均为ver1,未被修改的均是ver0(省略不写)\\
%来源:\begin{asparaitem}
%      \item 若来源词条与0稿中词条冲突,则按照来源词条为准\\
%      \item 若来源词条相互冲突,以时间长的为准,若内容上有补充,则互补\\
%      \item 1BJB——笔记本,此指2011-05-05前(含05-05)的BJB内容\\
%      \item 2XZH——此指2011-05-05以后上课老师所说内容\\
%      \item 3myself——此指方法大全的非关键部分的修订\\
%      \end{asparaitem}
%时间:修订时分两天进行,分别按来源稿和0稿进行对比,时间均为2011-05-24或2011-05-25,而与来源稿实际时间无关\\
%注:若以后发现笔记本内容或XZH口述内容(2011-05-25前含2011-05-25)与0稿不同而又未修改,则修改后标记上ver1+来源+时间\\
%若XZH当天口述内容,并发现笔记本上内容与0稿不同,则依然标上ver1+XZH+时间\\
%时间以修改本文当时间为准,而与实际获取内容时间无关!\\
Version4.0\\
2011-05-26修改\\%修订五:说明顺序,说明文语言准确性,说明文语言准确性
Version4.5\\
2011-06-14修改部分内容\\%并且改为latex
Version5.0\\
2011-06-22无内容\\%latex的新定义\hatsection.\hatsubsection
Version5.1\\
2011-07-09说明方法作用重新整理,论证方法作用增加换行\\
Version5.2\\
2011-07-10无内容\\%可能是代码修改
Version5.3\\
2011-07-11无内容\\%可能是代码修改
Version5.4\\
2011-07-14修改议论文整体感知\\
Version5.5\\
2011-07-16无内容\\%可能是代码修改
Version5.6\\
2011-07-16无内容\\%可能是代码修改
Version5.7\\
2011-07-17无内容\\%重新调整分段、断行
Version5.8\\
2011-07-19增加,修改\\
Version5.9\\
2011-07-31编号修改\\
Version6.0\\
2011-08-14添加、修改、目录颜色修改(特此感谢陆文昊)\\
-----------------------------------What's new----------------------------------\\
文言文部分\\
翻译古文,首要的是要掌握“补”、“顺”、“选”、“活”四种方法。\\
补——补全成分:主语、宾语、介词、量词等成分,使其符合现代汉语语言习惯\\
调——把古汉语倒装句调整为现代汉语句式。\\
选——一词多义翻译时,必须结合语言环境和事理逻辑选择合适的义项。\\
活——对于一些古今异义词要用现代词汇替换。\\
(尽可能字字落实)
\\
本文档分为五个部分\\
五个部分\&页数信息\\
概述<formal>\pageref{abstractformal}\\
概述<informal>\pageref{abstractinformal}\\
目录(main)\pageref{contents}\\
正文(main)\pageref{main}\\
附录<formal>(appendix)\pageref{appendixformal}\\
附录<informal>(appendix2)\pageref{appendixinformal}\\
\\
注:若显示为??则说明该文档不包括这一内容\label{appendixformal} \\
\\ 