% # -*- coding: utf-8 -*-
\typeout{now loading appendix.tex}
\appendix
\pagenumbering{Roman}
\section{附录}
文言文部分\\
翻译古文,首要的是要掌握“补”、“顺”、“选”、“活”四种方法。\\
补——补全成分:主语、宾语、介词、量词等成分,使其符合现代汉语语言习惯\\
调——把古汉语倒装句调整为现代汉语句式。\\
选——一词多义翻译时,必须结合语言环境和事理逻辑选择合适的义项。\\
活——对于一些古今异义词要用现代词汇替换。\\
(尽可能字字落实)
\\
立意部分\\
环保:保护绿色植物,保护生态平衡,注重可持续发展,注重发展和谐环境,控制人口增长;\\
老师:关心学生成长,工作细致,注意教育方法,善于保护孩子的自尊心\\
失败:失败能丰富人的阅历,还能锻炼人的意志,增加人的才干,从失败中吸取教训,有利于自己走向成功\\
本文档分为五个部分\\
五个部分\&页数信息\\
概述\pageref{abstractformal}\\
目录(main)\pageref{contents}\\
正文(main)\pageref{main}\\
附录(appendix)\pageref{appendixformal}\\
\label{appendixformal} \\
贡献者(除语文老师外)<学号>:41,6,32,11,47