% # -*- coding: utf-8 -*-
\pagenumbering{roman}
\tableofcontents
\newpage
\pagenumbering{arabic}
\section{现代文知识(概念)}
\hatsubsection{句子根据语气分:}
  陈述句、疑问句、祈使句(请求、命令、期望、劝阻、催促)感叹句
\hatsubsection{语言表达方式:}
  叙述、描写、议论、抒情、说明
\hatsubsection[\hatx{ver1byXZH 2011-05-24}]{修辞方法:}
  设问(问号)、反问(问号)、排比(三者)、对比、比喻、拟人、夸张(夸大或缩小)、引用
\hatsubsection[\hatb{ver1byBJB 2011-05-24}]{描写(细分):}
  人物:肖像(外貌、神态)、动作、语言、心理;
  环境:自然、社会
\hatsubsection{记叙顺序:}
  顺叙、倒叙、插叙(中间的段落)
\hatsubsection{说明顺序:}
  时间、空间、逻辑
\hatsubsection[\hatb{ver1byBJB 2011-06-14}]{说明方法:}
列数字、举例子、作比较、列图表;下定义、引资料、分类别、打比方
\hatsubsection[\hatb{ver1byBJB 2011-05-24}]{论据的种类:}
\begin{description}
\item [事实论据](有代表性的确凿的事例,历史史实,统计数字等)(事例史实以及数据)
\item [理论论据](经典著作,权威性言论,名人名言,警句,言语,俗语,自然科学的原理、定律、公式等)
\end{description}
\hatsubsection[\hatx{ver1byXZH 2011-06-14}]{论证方法:}
例证法(举例论证法)、引证法(引用论证法)、对比论证、比喻论证、类比论证
\hatsubsection{论证结构(思路结构):}
  总分式、并列式、层进式、对照式
\hatsubsection[\hatx{ver1byXZH 2011}]{小说三要素:}
\begin{description}
\item[小说]小说的中心,反映文章主题
\item[环境]分为自然环境、社会环境
\item[故事情节]故事的开端、发展、高潮、结局
\end{description}
\section{现代文综合知识(方法)}
\hatsubsection{词的感情色彩:}
  褒义贬用(讽刺了\ldots{}\ldots{});贬义褒用(使语言幽默生动)
\hatsubsection[\hatb{ver2by6BJB 2011-06-13}]{词序(词或词组能否互换):}
\begin{enumerate}
\item 逻辑关系(大多数):\ldots{}\ldots{}(前者)是\ldots{}\ldots{}(后者)的前提(或\ldots{}\ldots{}(后者)在\ldots{}\ldots{}(前者)的基础上才能形成,或先有(前者)\ldots{}\ldots{},才有\ldots{}\ldots{}(后者));符合其发生、发展逻辑顺序(或符合人们认识事物的客观规律)
\item 强调作用:突出了\ldots{}\ldots{}(极少)
\end{enumerate}
\hatsubsection[\hatm{ver1 bymyself 2011-06-14}]{选择词语:}
  一种是选择能表现人物思想性格的词语(围绕``表现人物的思想性格''来回答);另一种按照``辨析近义词''的方法进行
\hatsubsection[\hatb{ver1byBJB 2011-05-25}]{句子的含义:}
  表层:字面(扣住语境);深层:深刻、含蓄、双关、暗示等
\hatsubsection{句子在文中的作用:}
\begin{enumerate}
\item 句式特点(结合文体):记叙文——铺垫、伏笔、照应、过渡;描写、修辞、设置悬念等;说明文——过渡、照应;说明方法等;议论文——过渡、照应;论据、论证方法等)
\item 表现中心(包括一部分文字的中心)
\end{enumerate}
(结尾处的作用有内容上:点明中心、深化主题:结构上照应开头、点题等)
\\注意:过渡通常有两种情况:承上启下和引出下文
\hatsubsection{选择句子:}
依据句式特点进行:修辞句、描写句;哲理句、倒装句(句子成分、分句)..
(分号后面的少)
\hatsubsection{句序(语句能否互换):}
\begin{enumerate}
\item 过渡作用;
\item 照应作用:
\item 主次关系:
\item 总分关系
\end{enumerate}
\hatsubsection{填写句子:}
\begin{enumerate}
\item 填写承上启下的过渡句(包括起过渡作用的设问句);
 \item 填写总起句(中心句)
\end{enumerate}
\hatsubsection{反问句与陈述句谁好:}
\begin{enumerate}
\item 反问句能表达更加强烈语气,强调了\ldots{}\ldots{}(该句要表达的意思):
\item 写出这两句加强(削弱)了\ldots{}\ldots{}语气(①居多)
\end{enumerate}
\hatsubsection{段序(段与段能否互换):}
\begin{enumerate}
\item 过渡关系(承上启下):
\item 连接关系(与上下段的关系);
\item 总分关系;
\item 照应关系;
\item 递进关系;
\item 主次关系;
\end{enumerate}
\hatsubsection{段落在文中的作用:}
  同``5的句子在文中的作用'';记叙文中加考虑是否``插叙''
\hatsubsection{指代:}
  往往在它的前面,往往原文中有现成的;代入法验证
\hatsubsection[\hatb{ver2by6BJB 2011-06-14}]{设问(设问句)的作用:}
\begin{description}
\item[文中]引起读者的注意和思考,使文势有变化,在结构上起过渡作用,引出下文(,突出某些内容),使文章结构紧密,条理清楚。
\item[标题]引起读者的注意和思考,能更好的表现\ldots{}\ldots{}中心;增加读者的阅读兴趣
\end{description}
\hatsubsection{反问的作用:}
  表达更加强烈的语气,更加肯定的意思(要分析)
\hatsubsection[\hatb{ver2by6BJB 2011-06-13}]{排比的作用:}
\begin{enumerate}
\item 增强语言气势,强调或突出了\ldots{}\ldots{}(排比句想要表达的意思,即中心):
\item 分别从几个角度描绘(说明、论证)了\ldots{}\ldots{}(较少)
\end{enumerate}
\hatsubsection{对比的作用:}
  将\ldots{}\ldots{}与\ldots{}\ldots{}作对比,突出了\ldots{}\ldots{}
\hatsubsection{比喻(拟人):}
  记叙文,生动形象地描写了\ldots{}\ldots{}(内容),从而表现了\ldots{}\ldots{}(中心);说明文:生动形象地说明了\ldots{}\ldots{}(使说明的内容通俗易懂);议论文:生动形象地论证了\ldots{}\ldots{}
\hatsubsection{夸张的作用:}
  强调了\ldots{}\ldots{},(有的还要``从而表现了\ldots{}\ldots{}'')
\hatsubsection{语言表现力(赏析):}
  运用了\ldots{}\ldots{}(特色:①修辞手法;②描写方法;③其它\ldots{}\ldots{})+(具体)生动形象地写出了\ldots{}\ldots{}(如人物的动作、神态、心理活动、景物怎样的特征等)(,为下文的\ldots{}\ldots{}作铺垫)+从而表现出(或说明了、论证了)\ldots{}\ldots{}(对表达文章中心起的作用)
  注:仅在有细节描写时考虑``具体''
      仅记叙文中考虑``,为下文的\ldots{}\ldots{}作铺垫''
  ``其它''指:褒义贬用/贬义褒用、动词/形容词连用、拟声词、叠词;倒装句、整句和散句等
\hatsubsection{冒号用法:}
\begin{enumerate}
\item 提示下文;
\item 总结上文
\end{enumerate}
\hatsubsection{引号用法:}
\begin{enumerate}
\item 表示引用;
\item 强调,着重指出;
\item 表示特殊含义(或特定的称谓);
\item 反语,表示讽刺或否定
\end{enumerate}
\hatsubsection{破折号的用法:}
\begin{enumerate}
\item 补充、解释说明;
\item 意思的转折(话题的转换);
\item 表示意思的递进;
\item 表示语音的延长或停顿、中断
\end{enumerate}
\hatsubsection{省略号用法:}
\begin{enumerate}
\item 表示列举的省略;
\item 表示内容的省略;
\item 表示说话断断续续或含含糊糊、支支吾吾
\end{enumerate}
\hatsubsection{书名号用法:}
  书名、报名、刊名、篇名、剧名等
\section{记叙文}
\hatsubsection{记叙文整体感知:}
  目标:写了\ldots{}\ldots{};表现了(赞美或揭露了;揭示了哲理)\ldots{}\ldots{}(有时还有表达了\ldots{}\ldots{}思想感情)\\
  方法:\begin{enumerate}\item 看题目(中心内容/思想意义);
        \item 划出评价人或物的词句;
        \item 划出议论句、抒情句、哲理句;
        \item 看开头与结尾(尤其是看(划)结尾):
        \item 内容与细节描写处;
        \item 题干
\end{enumerate}
\hatsubsection{词语在文中的含义;}
结合语言环境作具体解释;少数结合其在古汉语中的意思
\hatsubsection{是否矛盾:}
``殊途同归'':从两个角度分别叙述,(有时要写明是为了更好地表现中心)
\hatsubsection{概括:}
一般由``谁''+``做了什么''进行表达(但不是字越少越好)
\hatsubsection{``具体表现'':}
  在文中可以找到
\hatsubsection{点评语句:}
  或从内容(包括中心)方面,或从语言特点上进行
\hatsubsection{顺叙的作用:}
  可以使事情的来龙去脉清晰地表现出来
\hatsubsection{倒叙的作用:}
  使结构有变化,叙述有波澜,以制造悬念,引人入胜;对比(较少)
\hatsubsection[\hatb{ver2by6BJB 2011-06-14}]{插叙的作用:}
  补充交待了\ldots{}\ldots{}(概括这部分内容)(有的有``为下文的\ldots{}\ldots{}作铺垫'')+更好地表现了\ldots{}\ldots{}(结合``中心''分析)+使文章更充实,内容更生动,主旨更突出(人物形象更鲜明)
\\注:仅在中间段落考虑插叙
\hatsubsection{人物描写的作用:}
  刻画其思想品质和性格特征(有时表现人物的思想感情)
  \\侧面描写:为正面插写服务,即能更好地刻画人物的形象
  \\细节描写:刻画人物思想品质和性格特征
  \\方法:生动形象地写出了\ldots{}\ldots{}(内容),从而表现出\ldots{}\ldots{}(中心)
\hatsubsection{补充心理描写:}
  常见的有表达``感激、愧疚\ldots{}\ldots{}或矛盾的心理''等;要能连接上下文:不能太短
\hatsubsection[\hatx{ver1byXZH 2011-05-24}]{环境描写的作用:}
\begin{enumerate}
\item 交代故事发生的时间、地点(或背景)(一般在开头);
\item 渲染\ldots{}\ldots{}气氛,营造\ldots{}\ldots{}氛围;
\item 衬托\ldots{}\ldots{}心情(或思想感情);
\item 为下文\ldots{}\ldots{}作铺垫(较少有``与下文\ldots{}\ldots{}相照应''、``与文中\ldots{}\ldots{}作对比'');
\item 推动故事发生的情节(非头尾段);
\item 刻画人物思想性格;
\item 表现主旨,突出文章中心
\end{enumerate}
\hatsubsection{详写与略写:}
  依据``能否更好地表现中心思想''来回答
\hatsubsection{结尾部分的作用:}
  结构:\begin{enumerate}\item 照应开头;
        \item 点明标题;
  内容:\item 突出主旨;
        \item 深化主题(有时还有含蓄生动,耐人寻味,给予读者以想象等)\end{enumerate}
        (深化主题:通常有两种①揭示深刻的哲理;②其品质(精神)影响到他人。)

\hatsubsection{标题的含义:}
\begin{enumerate}
\item 浅层(与中心内容有关);
\item 深层(与思想意义有关)
\end{enumerate}
\hatsubsection{标题的作用:}
\begin{enumerate}
\item 写作特点\\
    三种文体都有:``运用比喻的修辞手法''、``运用设问的修辞手法''.\\
    单记叙文有的:``设置贯穿全文的线索''、``设置悬念,引人入胜''。(很少有``与文中\ldots{}\ldots{}内容作对比''):\\
\item 深层(与思想意义有关)。(都要进行分析)
\end{enumerate}
\hatsubsection[\hatb{ver2by6BJB 2011-06-14}]{写作特点:}
  线索、悬念、倒叙、插叙、详略、抑扬、对比、照应、开头、结尾、衬托、以小见大和象征、联想、想象、托物言志、借物抒情等;综合运用多种修辞手法、综合运用多种描写方法等\\
 ``线索''通常有两种:\begin{enumerate}\item 围绕线索写了若干个情节(一般是一件事);(《笑》、《沉船之前》)
                      \item 围绕线索写了若干件事(材料、片段)(《故乡》)\end{enumerate}
  另一种分类:\begin{enumerate}\item 以时间为线索:《沉船之前》
              \item 以人物行踪为线索:《故乡》
              \item 以人物见闻为线索:《孔乙己》
              \item 物:《小桔灯》
              \item 感情变化:《荔枝蜜》
              \item 事件:《最完美的礼物》
              \item 心理活动:《最后一课》、《我不是懦夫》\end{enumerate}
线索的好处:使结构更严谨,中心更突出
\hatsubsection{人物品质:}
  往往用双音节词或多音节词组表达
\hatsubsection{思想感情:}
  常见的有``感激、感恩、赞美、敬意、敬仰、爱国、思念、喜爱、喜悦、愧疚、懊恼、后悔、沮丧、忏悔、惆怅、悲伤、悲凉''等
\hatsubsection[\hatm{ver1bymyself 2011-05-24}]{记叙文语言特点:}
  生动、形象(方法同``二/19语言表现力'')、具体(分析``描写'')
\section{议论文}
\hatsubsection[\hatx{ver2byXZH 2011-06-14}]{议论文整体感知:}
  目标:找出或概括论点(部分文字是``分论点'')\\
  (``论点''与``分论点''统称为``观点'')\\
  方法:\begin{enumerate}\item 看题目(论题式、论点式);
        (论题式:论点应包含论题部分,并对其有看法、见解或主张等;论点式:找到论点或标题出现的地方推敲其是论点的一部分还是全部;若找不到,一般该标题便是论点。)(标题不宜长)
        (论点、分论点一般不能是疑问句、比喻句,而应是陈述句)
        \item 圈出标志性语言(总之、可见、还、也、但是\ldots{}\ldots{});
        \item 划出主体段落开头与结尾的中心句(总起句、总结句、过渡句);(一般即为第一/倒数第一句,但是也有例外)
        \item 看思路结构(总分式、并列式、层进式、对照式):
        \item 看开头与结尾;
        \item 看论据(但有的是证明分论点)
\end{enumerate}
\hatsubsection[\hatb{ver2by6BJB 2011-06-14}]{论证结构(思路结构)与论点的关系:}
\begin{enumerate}
\item 总分式:论点往往在开头或结尾(包括:分总、总分总)
\item 并列式:论点是分论点相加(合成)
\item 层进式:论点往往在结尾
\item 对照式:看其强调的是什么
\end{enumerate}
如果是部分文字的论证结构(思路结构),其与论点的关系不存在
\hatsubsection[\hatx{ver2byXZH 2011-06-14}]{缘``事''而发:}
——仅指可能出现在议论文(杂文)的第一段开头(有时不止一段),还未进行议论(``发'')的文字。
  一定有的答案:①引出论题,②引出论点(①②中至少有一者,且在缘事而发的下一节前必须出现),③作为论据证明论点,④引出论题进而引出论点(并证明它);
  不一定有的答案:增加文章的趣味性,引起读者的阅读兴趣
\hatsubsection[\hatb{ver1byBJB 2011-05-24}]{论据的作用:}
  运用\ldots{}\ldots{}事实论据(引用\ldots{}\ldots{}理论论据),有力地论证了\ldots{}\ldots{}(遵循``先前后后,先近后远,先原文后概括''进行)
  (有些``名人逸事''有``增加读者阅读兴趣''的作用(少))
\hatsubsection[\hatx{ver2byXZH 2011-06-14}]{论证方法的作用:}
\begin{enumerate}
\item 例证法(举例论证法)
\item 引证法(引用论证法)
答案格式:运用\ldots{}\ldots{}论证方法,列举\ldots{}\ldots{}事例(引用\ldots{}\ldots{}名言),有力地论证了\ldots{}\ldots{}(遵循``先前后后,先近后远,先原文后概括''进行)
\item 对比论证
答案格式:分别列举了\ldots{}\ldots{}事例和\ldots{}\ldots{}事例(将\ldots{}\ldots{}事例和\ldots{}\ldots{}事例作对比),从正反两方面有力地论证了\ldots{}\ldots{}(遵循``先前后后,先近后远,先原文后概括''进行)
\item 比喻论证
答案格式:把\ldots{}\ldots{}比作\ldots{}\ldots{},生动形象地论证了\ldots{}\ldots{}(遵循``先前后后,先近后远,先原文后概括''进行)
\item 类比论证
\end{enumerate}
\hatsubsection{论据能否删、换等:}
\begin{enumerate}
\item 能否证明论点(分论点);
\item 是否照应(往往有``古今中外'');
\item 是否一正一反(正反对比论证),使论据更充分;
\item 论据的广泛性(不同的国籍、不同的朝代、不同的领域等)
\end{enumerate}
\hatsubsection[\hatb{ver1byBJB 2011-05-24}]{补充(概括)论据:}
  人名(事件名)+紧扣论点(或分论点)+结果
\hatsubsection[\hatm{ver1bymyself 2011-06-14}]{论证过程(怎样进行论证)(又称文章结构):}
\begin{enumerate}
\item 提出问题(``提出\ldots{}\ldots{}论题''或``提出\ldots{}\ldots{}观点'')、很少有``通过事例或现象引出议论的问题'');
\item 分析问题
依据以下三个方面进行
\begin{description}
\item[论证结构]总分式/并列式/层进式/对照式
\item[论据]事实论据/理论论据(证明论点或分论点)
\item[论证方法]举例论证/引用论证/对比论证;
\end{description}
\item 解决问题(提出论点或结论、提出应对措施等)
\end{enumerate}
注:一部分文字可能没有``提出问题''、``解决问题'',但一定有``分析问题''
\\做法:先用``//''将提出问题、分析问题和解决问题区分开,再用``/''将分析问题部分依据上述方法分成若干层。
\hatsubsection{严密性:}
\begin{enumerate}
\item 语言:有无的区别点+(有的有``<不>符合事实''或``避免说法绝对化<片面性>'')+能更好地表现中心
\item 论证:(思路)生活中既有文中重点阐述的\ldots{}\ldots{},又有它的另一种\ldots{}\ldots{}现象,这样可避免论证的绝对化<片面性>,从而使得论证更加严密,论点更有说服力。
\item 结构(段与段能否互换):前面的方法+<不>能体现(议论文)语言严密性
\end{enumerate}
\hatsubsection{生动性:}
  方法同记叙文的``语言生动性''的做法
\section{说明文}
\hatsubsection[\hatx{ver2byXZH 2011-06-14}]{说明文整体感知:}
目标:找出或概括说明对象及其特征\\
(注意是事物性说明文还是事理性说明文)\\
(注:找特征的用途:特征即为中心思想或段落大意)\\
(有的还要考虑``表现民间工艺的精湛''或``表达对祖国河山的赞美之情''等)\\
方法:\begin{enumerate}\item 看题目(说明对象;说明对象的特征);
        \item 关注思路结构(特别关注``总分式''、``并列式'');
        \item 看开头与结尾(中心在头尾居多);
        \item 圈出路标性的词语;(``但是''、``而''、``同时''、``又''、``而且''、``也''、``可见''、``所以''、``另外''、``然而''以及类似于``一是''、``二是''之类的表明顺序的词语);
        \item 划出主体段落开头与结尾的中心句(总起句、总结句、过渡句);
        \item 看说明方法(要注意的是有的说明方法说明说明对象的``部分特征'')\end{enumerate}
\hatsubsection[\hatx{ver2byXZH 2011-06-14}]{说明方法的作用:}
运用了\ldots{}\ldots{}说明方法,\\
\[列举\ldots{}\ldots{}的例子^{举例子}(数据^{列数字}),列出了\ldots{}\ldots{}的图(数据、模型)^{列图表},将\ldots{}和\ldots{}作比较^{作比较},\]
\[具体^{举例子、列图表}(准确)^{列数字、列图表}、(强调、突出)^{作比较}、(一目了然、清晰)^{列图表}\]
地说明了\ldots{}\ldots{}(遵循``先前后后,先近后远,先原文后概括''进行)\\
\xout{详细:}\\
\xout{举例子——具体}\\
\xout{列数字——准确}\\
\xout{作比较——强调、突出}\\
\xout{列图表——具体+一目了然+准确+清晰}\\
``主要说明方法'':写一种(依据:①包含与被包含;②占的文字范围大的)
\hatsubsection[\hatx{ver1byXZH 2011-05-26}]{说明顺序:}
时间顺序(以事物发展的时间先后顺序);空间顺序(由表及里,从上到下,从前到后,从外到内,由远及近,从整体到部分);逻辑顺序(由表及里,由浅入深,由具体到抽象,从现象到本质,从简单到复杂,由主要到次要,由性质到功能,由原因到结果、由一般到个别、由整体到局部等)
(``傻瓜答题法'':不是空间顺序、时间顺序,便是逻辑顺序)
\hatsubsection[\hatm{ver2bymyself 2011-06-14}]{说明文语言特点:}
\hatsubsubsection[\hatx{ver2by6BJB 2011-06-14}]{说明文语言准确性:}
\begin{enumerate}
\item 删去某词(词组)语意起变化:变化之处+<不>符合事实(避免说法绝对化<片面性>)+<不>能体现(说明文)语言准确性;
\item 删去某词(词组)语意未变化:(它在文中加强语气,)强调了\ldots{}\ldots{}(这句话想要表达的意思),能体现(说明文)语言准确性;
\item 约数:理解(常见的是``符合人们的认知能力''或``因为其是动态的\ldots{}\ldots{}'')+<不>符合事实+<不>能体现(说明文)语言准确性
\end{enumerate}
\hatsubsubsection[\hatb{ver2by6BJB 2011-05-26}]{说明文语言生动性:}
参照记叙文的``语言生动性''的做法:运用\ldots{}\ldots{}修辞手法(描写方法),生动形象地说明了\ldots{}\ldots{}(中心)(``比喻''有时要写``使说明的内容通俗易懂'')
\hatsubsection[\hatx{ver1byXZH 2011-06-14}]{运用传说(有趣故事,古诗文,谚语,史实等):}
\begin{enumerate}
\item 能说明事物的特征(包括``分特征'');
\item 增加文章的趣味性,引起读者的阅读兴趣
\item 引出下文说明对象及其特征
\end{enumerate}
\hatsubsection[\hatx{ver2bymyself 2011-06-14}]{说明文的分类:}
分类一
\begin{description}
\item[平实性说明文]仅有说明(体现了说明文语言的准确性)
\item[生动性说明文]除说明外还有描写或修辞(体现了说明文语言的生动性)
\end{description}
分类二
\begin{description}
\item[事物性说明文]
\item[事理性说明文]
\end{description}

注意:没有``方法''的阅读题目应``紧扣文章中心(思考其与中心(包括一部分文字的中心)的关系),紧扣语言环境(思考其与上下文的关系)''。