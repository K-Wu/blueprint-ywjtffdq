% # -*- coding: utf-8 -*-
%%include
%%beta,stable,custom
%%abstract2,main,appendix,appendix2
%%info.png,readme.jpg
\documentclass{article}


%中文支持宏包
\usepackage{xeCJK}
\setCJKmainfont[BoldFont=SimHei]{YouYuan}
\usepackage{xunicode}
\usepackage{xltxtra}
\XeTeXlinebreaklocale"zh"
\XeTeXlinebreakskip= 0pt plus 1pt
%中文支持宏包
\renewcommand{\baselinestretch}{1.38}
\setlength{\parindent}{0em} % 设置长度.这里指设置段首缩进的长度.2em指两个字.
%\usepackage{hyperref}
%\hypersetup{colorlinks=true}
\usepackage{hyperref}
\hypersetup{colorlinks=true}
\hypersetup{linkcolor=black,anchorcolor=black,citecolor=black}
\usepackage{graphicx}
\usepackage{xcolor}
\usepackage{ulem}
\usepackage[left=1.6cm,right=2cm,top=1.6cm,bottom=2cm]{geometry}
\usepackage{ifthen}
\usepackage{paralist}


%把subsection改成subsubsection风格
\makeatletter
\renewcommand\subsection{\@startsection{subsection}{2}{\z@}%
                                     {-3.25ex\@plus -1ex \@minus -.2ex}%
                                     {1.5ex \@plus .2ex}%
                                     {\normalfont\normalsize\bfseries}}

%创建自定义概述
\newcommand\hatabstractname{Abstract}
\if@titlepage
  \newenvironment{hatabstract}{%
      \titlepage
      \null\vfil
      \@beginparpenalty\@lowpenalty
      \begin{center}%
        \bfseries \abstractname
        \@endparpenalty\@M
      \end{center}}%
     {\par\vfil\null\endtitlepage}
\else

%创建自定义概述
  \newenvironment{hatabstract}{%
      \if@twocolumn
        \section*{\hatabstractname}%
      \else
        \small
        \begin{center}%
          {\bfseries \hatabstractname\vspace{-.5em}\vspace{\z@}}%
        \end{center}%
        \quotation
      \fi}
      {\if@twocolumn\else\endquotation\fi}
\fi
\makeatother



\newcommand{\hatsubsection}[2][\hatn{ver0}]{\subsection{\texorpdfstring{#2#1}{#2}}\label{#2}}
\newcommand{\hatsubsubsection}[2][\hatn{ver0}]{\subsubsection{\texorpdfstring{#2#1}{#2}}\label{#2}}
\newcommand{\superref}[1]{#1(在\pageref{#1}页的\ref{#1}上)}


%经典配色方案
%\newcommand{\hatb}[1]{\colorbox{blue!100}{#1}}
%\newcommand{\hatx}[1]{\colorbox{yellow!100}{#1}}
%\newcommand{\hatm}[1]{\colorbox{green!100}{#1}}
%\newcommand{\hatn}[1]{\colorbox{violet!100}{#1}}


%小清新配色方案
%\newcommand{\hatb}[1]{\colorbox{orange!100}{#1}}
%\newcommand{\hatx}[1]{\colorbox{yellow!100}{#1}}
%\newcommand{\hatm}[1]{\colorbox{cyan!100}{#1}}
%\newcommand{\hatn}[1]{\colorbox{lightgray!100}{#1}}

%黑白配色方案
\newcommand{\hatb}[3]{\colorbox{gray!75}{ver#2by#1BJB #3}}
\newcommand{\hatx}[2]{\colorbox{gray!75}{ver#1byXZH #2}}
\newcommand{\hatm}[3]{\colorbox{gray!25}{ver#2by41 #3 #1}}
\newcommand{\hatn}[1]{\colorbox{gray!25}{#1}}

%从一下开始所有未注明不可删除注释全删除

%\newcommand{\noa}{X}
%\newcommand{\BA}{NI}
\newcommand{\nob}{Abstract}
\newcommand{\nobhat}{说明}
\newcommand{\BB}{Beta}
\newcommand{\noc}{\par 此处的空白恰到好处,正是本文档的精华之处}
\newcommand{\noticewave}{\par 留意波浪线\par 加波浪线内容未确定}
\newcommand{\BC}{\par 这是一个不稳定的版本}

%此三行不可被删除
%\newcommand{\noprint}{\par 此份文档不被用来打印。\par This file/document is not used to print.}
%\newcommand{\origintal}{\par 此份文档是原始的。\par This file/document is ORIGINTAL.}


%\newcommand{\BD}{appendix}
%\newcommand{\nod}{appendix}

%\newcommand{\BE}{Undefined}
%\newcommand{\noe}{Undefined}

\newcommand{\zuozhe}{Undefined}


%\newcommand{\noehat}{HAWX AT TEAM\thanks{一份完整的致谢名单在A.3致谢PAGE\pageref{thanks}上}}
%\newcommand{\BF}{\today}
%\newcommand{\nof}{\today}

%\newcommand{\nofhat}{May 2011}



%\newlength{\seplength}
%\settowidth{\seplength}{*}
%\newcommand{\seperator}{\whiledo{\lengthtest{\linewidth>\seplength}{\linewidth-\seplength-1mm} *}}


\typein[\whatfiletoinput]{0.type in a configuration.``stable''``beta''``custom''}
\input{\whatfiletoinput}
