% # -*- coding: utf-8 -*-
\newpage
\appendix
\pagenumbering{Roman}
\section{附录}
\subsection{说明}
\begin{verbatim} --------------------------------------来源-------------------------------------
参见References
------------------------------------Lai Yuan-----------------------------------
------------------------------------修订笔记-----------------------------------
*************************************按页分************************************
第一页:从一/1~二/5
第二页:从二/5~二/17
第三页:从二/18~三/4
第四页:从三/5~三/16
第五页:从三/16~四/5
第六页:从四/16~五/4
第七页:从五/5~注意
************************************An Ye Fen**********************************

************************************按结构分***********************************
***********************************详见plus***********************************
一、现代文知识(概念)(共10)(原稿中所有的序号都被圈了)
二、现代文综合知识(方法)(共24)(原稿中“5句子在文中的作用”“10段序”前序号被圈)
三、记叙文(共20)
四、议论文(共8)(“6补充论据”后手写“(概括)”)
五、说明文(共6)(“1说明文整体感知”目的中手写添加一部分内容)
注意
*********************************An Jie Gou Fen********************************
---------------------------------Xiu Ding Bi Ji--------------------------------

-------------------------------------修订历史----------------------------------
Version1
2011-05-05扫描,老师仅重新修订了说明文部分,
Version2
2011-05-06修改部分错误,根据老师上课讲的内容补充说明文部分(主要是整体感知中的路标性词语),并增加“目录”内容
2011-05-15高级添加(线索、文言文翻译)
2011-05-19 五、1“说明文整体感知”细微调整,说明文和议论文部分的几个括号的修正
Version3
2011-05-20一、4改为大括号,补充四、1“议论文整体感知”,补充一、8“论据的种类”,修改议论文部分(主要是论据种类、论证方法、论证结构以及他们的答题格式),修改二、19“语言表现力(赏析)”,删除文言文部分
Version3.1
2011-05-24从学校偷来XZHU盘全部课件,并增加了文言文部分
2011-05-24~2011-05-25增加修改标记并复刻原稿和原稿plus(即加上xzh亲笔来源于2011-05-05)此次修改标记的添加规则如下:
以原稿plus为ver0,进行一一对比(下称0稿)
添加字符由三个部分组成:
时间、来源、版本
版本:此次版本号凡是被修改的均为ver1,未被修改的均是ver0(省略不写)
来源:若来源词条与0稿中词条冲突,则按照来源词条为准
      若来源词条相互冲突,以时间长的为准,若内容上有补充,则互补
      1BJB——笔记本,此指2011-05-05前(含05-05)的BJB内容
      2XZH——此指2011-05-05以后上课老师所说内容
      3myself——此指方法大全的非关键部分的修订
时间:修订时分两天进行,分别按来源稿和0稿进行对比,时间均为2011-05-24或2011-05-25,而与来源稿实际时间无关
注:若以后发现笔记本内容或XZH口述内容(2011-05-25前含2011-05-25)与0稿不同而又未修改,则修改后标记上ver1+来源+时间
若XZH当天口述内容,并发现笔记本上内容与0稿不同,则依然标上ver1+XZH+时间
时间以修改本文当时间为准,而与实际获取内容时间无关!
Version4.0
2011-05-26修订五:说明顺序,说明文语言准确性,说明文语言准确性
Version4.5
2011-06-14改为latex,并且修改部分内容
Version5.0
2011-06-22latex的新定义\hatsection.\hatsubsection
-----------------------------------What's new----------------------------------
文言文部分
翻译古文,首要的是要掌握“补”、“顺”、“选”、“活”四种方法。
补——补全成分:主语、宾语、介词、量词等成分,使其符合现代汉语语言习惯
调——把古汉语倒装句调整为现代汉语句式。
选——一词多义翻译时,必须结合语言环境和事理逻辑选择合适的义项。
活——对于一些古今异义词要用现代词汇替换。
(尽可能字字落实)
\end{verbatim}
本文档分为四个部分\\
四个部分\&页数信息\\
概述(abstract)\pageref{abstract}\\
目录(main)\pageref{contents}\\
正文(main)\pageref{main}\\
附录(appendix)\pageref{appendix}\\
注:若显示为??则说明该文档不包括这一内容
\newpage
\subsection{声明}
\begin{description}
\item[HAWX AT TEAM]的入门作品2
\item[蓝本计划]YWJTFFDQ
\item[源码支持]\XeTeX \LaTeXe  \LaTeX  \TeX
\item[关键字符串]$\backslash$input
\item[关键字符串2]\% \# -*- coding: utf-8 -*-
\item[编译软件]\emph{Texworks}
\item[编译编码]\emph{UTF-8}
\item[编译方式]\emph{XeLaTeX+MakeIndex+BibTeX}
\item[中文支持]\emph{编译方式+XeLaTeX宏包}
\item[XeLaTeX宏包]\emph{xeCJK}
\item[所用宏包]\footnote{按加载顺序,更改顺序可能会编译出错!}
\begin{enumerate}
\item xeCJK
\item xunicode
\item xltxtra
\item hyperref
\item graphicx
\item xcolor
\item ulem
\item geometry
\end{enumerate}
\end{description}
\subsection{致谢}\label{thanks}
\begin{itemize}
\item \underline{lshort}\footnote{位于CTAN/info/lshort/chinese}
\item \underline{ctex.org}
\item \underline{Chinatex}
\item \underline{boj.5d6d.com}
\item \underline{新浪博客''LaTeX-学习园地''}\footnote{blog.sina.com.cn/wangzhaoli11}
\item \underline{LaTeX编辑部}\footnote{zzg34b.w3.c361.com}
\item \underline{百度知道用户''Chinatexer''}
\item \underline{Ubantu中文论坛}\footnote{forum.ubuntu.org.cn}
\item \underline{百度文库}\footnote{wenku.baidu.com}
\item \underline{lnote}\footnote{bbs.ctex.org/viewthread.php?tid=43774}
\item \underline{163博客''简单执着''}\footnote{chenli-0925.blog.163.com}
\item \underline{\TeX Cookbook}
\item \underline{QQ群TeX\&LaTeX社区-ChinaTeX}\footnote{群号码:91940767}
\item \underline{QQ群TeX\&LaTeX社区3群-ChinaTeX}\footnote{群号码:141877998}
\item \underline{QQ用户Clark Ma}\footnote{QQ号:1113706230}
\item \underline{《\LaTeXe 完全学习手册》}\footnote{清华大学出版社,latex编辑部}
\end{itemize}
在此,对上述网站/个人/书籍表示感谢!\\
HAWX AT TEAM\\
\today\\ %输入今日日期
\newpage
\begin{thebibliography}{99}
\bibitem{XZH的笔记}徐正华.语文解题方法大全.绝密文档
\bibitem{XZH的课件}徐正华.《黔之驴》.高级U盘
\bibitem{XZH}徐正华上课口述内容(2011-05-05后)
\bibitem{BJB}笔记本内容(2011-05-05前)
\bibitem{6BJB}沈同学的笔记本内容(2011-05-05前)
\bibitem{32BJB}刘同学的笔记本内容(2011-05-05前)
\end{thebibliography}
在此,对上述个人表示感谢!\\
HAWX AT TEAM\\
\today\\
The end. \label{appendix} 